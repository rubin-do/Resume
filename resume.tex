\documentclass[a4paper, 12pt]{article}
\usepackage{amssymb}
\usepackage{amsmath}
\usepackage[utf8]{inputenc}
\usepackage[russian]{babel}
\usepackage[OT1]{fontenc}
\usepackage{amsfonts}
\usepackage{ifthen}
\usepackage{tabularx}
\usepackage{textcomp}
\usepackage{physics}
\usepackage{listings}
\usepackage{color}
\usepackage{enumitem}
\usepackage{hyperref}
\usepackage{cancel}
\usepackage{mathtools}
\usepackage{graphicx}
\graphicspath{./images_3/}

\begin{document}
\begin{center}
    \LARGE
       \textbf{Исследование абстрактных циклических групп, ДЗ №1} \\ {\Large НИУ ВШЭ, Прикладная математика и информатика, 196 группа} \\ Рубин Даниил \\ \today.
\end{center}
\newpage

\section*{Исследовать абстрактную циклическую группу порядка 24}
    \subsection*{Запишем группу:}
        G = \{$1_G,x^{1}, x^{2}, x^{3}, x^{4}, x^{5}, x^{6}, x^{7}, x^{8}, x^{9}, x^{10}, x^{11}, x^{12}, x^{13}, x^{14}, x^{15}, x^{16}, x^{17}, x^{18}, x^{19}, x^{20}, x^{21}, x^{22}, x^{23}$\}
    \subsection*{Определим порядки всех элементов группы}
        \begin{itemize}
            \item O($1_G$) = 1
            \item O($x^1$) = 24
            \item O($x^{2}$) = 12
            \item O($x^{3}$) = 8
            \item O($x^{4}$) = 6
            \item O($x^{5}$) = 24
            \item O($x^{6}$) = 4
            \item O($x^{7}$) = 24
            \item O($x^{8}$) = 3
            \item O($x^{9}$) = 8
            \item O($x^{10}$) = 12
            \item O($x^{11}$) = 24
            \item O($x^{12}$) = 2
            \item O($x^{13}$) = 24
            \item O($x^{14}$) = 12
            \item O($x^{15}$) = 8
            \item O($x^{16}$) = 3
            \item O($x^{17}$) = 24
            \item O($x^{18}$) = 4
            \item O($x^{19}$) = 24
            \item O($x^{20}$) = 6
            \item O($x^{21}$) = 8
            \item O($x^{22}$) = 12
            \item O($x^{23}$) = 24
        \end{itemize}
            Тогда G = $<x^1>$ = $<x^5>$ = $<x^7>$ = $<x^{11}>$ = $<x^{17}>$ = $<x^{19}>$ = $<x^{23}>$
        \subsection*{Перечислим циклические подгруппы группы G}
            $H_1 = <x^2> = <x^{10}> = <x^{14}> = <x^{22}>$ $|H_1| = 12$\\
            $H_2 = <x^3> = <x^9> = <x^{15}> = <x^{21}>$ $|H_2| = 8$\\
            $H_3 = <x^4> = <x^{20}>$ $|H_3| = 6$\\
            $H_4 = <x^8> = <x^{16}>$ $|H_4| = 3$\\
            $H_5 = <x^{12}>$ $|H_5| = 2$\\
        \subsection*{Выпишем подгруппы}
            $H_1$ = \{$1_G, x^{2}, x^{4}, x^{6}, x^{8}, x^{10}, x^{12}, x^{14}, x^{16}, x^{18}, x^{20}, x^{22}$\}\\
            $H_2$ = \{$1_G, x^{3}, x^{6}, x^{9}, x^{12}, x^{15}, x^{18}, x^{21}$\}\\
            $H_3$ = \{$1_G, x^{4}, x^{8}, x^{12}, x^{16}, x^{20}$\}\\
            $H_4$ = \{$1_G, x^{8}, x^{16}$\}\\
            $H_5$ = \{$1_G, x^{12}$\}\\
        \begin{center}
            \includegraphics[width=10cm]{images_3/graph.png}    
        \end{center}
        
\end{document}